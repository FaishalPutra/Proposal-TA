% ==========================================
% BAB II STUDI LITERATUR
% ==========================================
\chapter{STUDI LITERATUR}
\label{chap:studi-literatur}
% \section{Penulisan Gambar, Tabel, Rumus, dan Kode}
% \lipsum[1]

\section{Mahasiswa dan Pengembangan Diri}
Menurut \Textcite{cunliffInstitutionalizingTransformativeLearning2018}, mahasiswa merupakan pembelajar yang sedang berada dalam proses transformasi cara pandang melalui pengalaman, refleksi kritis, dan dialog. Mahasiswa tidak hanya menerima teori secara pasif, tetapi juga mengembangkan cara berpikir baru melalui interaksi dengan pengalaman yang mereka alami serta proses refleksi yang menyertainya. hal ini sejalan dengan pemikiran \Textcite{deweyEXPERIENCEEDUCATION1938} yang menekankan bahwa pengalaman langsung merupakan komponen penting dalam proses belajar. Pembelajaran akan menjadi bermakna apabila mahasiswa terlibat secara aktif dalam aktivitas yang menghubungkan teori dengan praktik, sehingga mereka dapat membangun makna yang lebih dalam dari pengalaman tersebut. Melalui keterlibatan langsung, mahasiswa tidak hanya memperoleh informasi baru, tetapi juga belajar menafsirkan dan mengintegrasikan pengalaman tersebut ke dalam pemahaman mereka.

\Textcite{kolbExperientialLearningTheory2014} menegaskan bahwa mahasiswa mengembangkan pengetahuan melalui siklus pembelajaran yang meliputi pengalaman konkret, refleksi, konseptualisasi abstrak, dan eksperimen aktif. Hal ini menunjukkan bahwa pembelajaran yang efektif terjadi ketika mahasiswa terlibat dalam aktivitas yang sesuai dengan kapasitas dan kesiapan mereka, sehingga pengalaman yang diperoleh dapat diolah menjadi pengetahuan yang berkelanjutan. Pandangan tersebut sejalan dengan \textcite{eriksonErikEriksonsStages1950} yang menjelaskan bahwa individu pada rentang usia mahasiswa berada pada tahap \textit{identity versus role confusion}, yaitu fase ketika seseorang berusaha memahami siapa dirinya, apa kemampuan yang dimiliki, serta peran apa yang sesuai dalam lingkungan sosial maupun profesional. Pada tahap ini, mahasiswa membutuhkan pengalaman yang memungkinkan mereka mengeksplorasi minat, kompetensi, dan nilai-nilai pribadi agar dapat membentuk identitas diri.

kolaborasi di perguruan tinggi bukan sekadar aktivitas kerja kelompok, melainkan sebuah sistem pembelajaran sosial yang terstruktur. Dalam konteks tugas akhir ini, prinsip-prinsip kolaboratif dan kooperatif menjadi dasar dalam merancang platform kolaborasi mahasiswa yang memfasilitasi interaksi lintas jurusan, membantu proses pencarian partner berdasarkan keahlian, serta menciptakan ruang digital yang mendukung terbentuknya learning community yang produktif dan berkelanjutan.


\section{Kegiatan Kolaboratif}

Menurut \Textcite{roschelleConstructionSharedKnowledge1995}, kolaborasi merupakan aktivitas terkoordinasi yang muncul dari upaya bersama untuk membangun pemahaman yang selaras terhadap suatu permasalahan. Upaya kolaboratif akan menghasilkan pencapaian yang lebih tinggi dibandingkan kerja individu, karena memungkinkan terjadinya pertukaran gagasan, integrasi perspektif, dan peningkatan kemampuan interpersonal. \Textcite{dillenbourgWhatYouMean1999} menambahkan bahwa kolaborasi terjadi ketika individu bekerja bersama dalam keterlibatan yang saling tergantung untuk mencapai tujuan yang telah disepakati. Dalam konteks pendidikan tinggi kegiatan kolaboratif merupakan aktivitas pembelajaran yang melibatkan kerja sama antara dua individu atau lebih untuk mencapai tujuan bersama berupa proyek kelompok, lomba dalam tim, penelitian bersama, diskusi terstruktur, hingga aktivitas organisasi kemahasiswaan.


\section{\textit{Software Development Life Cycle} (SDLC)}

\textit{software development life cycle} (SDLC) merupakan kerangka proses yang digunakan untuk mengembangkan perangkat lunak secara sistematis, terstruktur, dan dapat dikendalikan. Menurut \textcite{sommervilleSoftwareEngineering2016}, SDLC memastikan bahwa setiap tahapan pengembangan mulai dari perumusan kebutuhan hingga pemeliharaan dilakukan secara konsisten untuk menghasilkan sistem yang memenuhi kebutuhan pengguna dan memiliki kualitas yang dapat dipertanggungjawabkan. SDLC secara umum mencakup beberapa fase inti seperti analisis kebutuhan, perancangan sistem, implementasi, pengujian, serta pemeliharaan. Pendekatan ini penting karena pengembangan perangkat lunak bukan hanya sekedar membangun fitur, tetapi juga mengelola risiko, mendokumentasikan keputusan, dan menjaga agar hasil akhir tetap sesuai dengan tujuan awal. 

\subsection{\textit{Waterfall model}}

\textit{waterfall model} adalah salah satu pendekatan SDLC paling klasik dan banyak digunakan dalam proyek yang memiliki kebutuhan stabil serta ruang lingkup yang terdefinisi dengan jelas. \Textcite{sommervilleSoftwareEngineering2016} menjelaskan bahwa \textit{waterfall} bekerja secara berurutan (sekuensial), di mana setiap fase harus diselesaikan sebelum beralih ke fase berikutnya. Pendekatan linear ini memberikan struktur yang kuat, dokumentasi yang lengkap, serta kontrol yang baik terhadap alur pengembangan.


\section{Sistem Rekomendasi dan pencocokan}

Menurut \textcite{ricciIntroductionRecommenderSystems2011}, sistem rekomendasi berfungsi sebagai alat pendukung keputusan yang menyaring informasi dan menampilkan pilihan yang paling sesuai bagi pengguna ketika mereka menghadapi banyak opsi. Sistem rekomendasi merupakan mekanisme yang dirancang untuk membantu pengguna dalam memilih item atau alternatif yang relevan berdasarkan preferensi, kebutuhan, atau karakteristik tertentu.

Konsep rekomendasi erat kaitannya dengan pencocokan, yaitu proses menilai sejauh mana dua entitas misalnya pengguna dan item memiliki kesesuaian berdasarkan atribut tertentu. \Textcite{burkeHybridRecommenderSystems2002} menjelaskan pencocokan sebagai inti dari sistem rekomendasi karena proses ini menentukan apakah sebuah item relevan bagi profil pengguna. Dalam konteks ini, pencocokan berfungsi untuk mengidentifikasi tingkat kesesuaian antara mahasiswa dan kegiatan kolaboratif yang tersedia. Proses ini penting karena setiap kegiatan memiliki tuntutan, peran, dan karakteristik tertentu, sementara mahasiswa memiliki minat, kemampuan, dan gaya kerja yang berbeda-beda. 

\subsection{\textit{Person Job Fit Theory}}

Konsep person job fit menjelaskan tingkat kesesuaian antara karakteristik individu dan tuntutan suatu pekerjaan atau aktivitas. Menurut \Textcite{kristof-brownCONSEQUENCESINDIVIDUALSFIT2005}, \textit{personjob fit} terjadi ketika kemampuan, keterampilan, minat, serta preferensi individu sejalan dengan kebutuhan, tugas, dan karakteristik suatu pekerjaan. Semakin tinggi tingkat kecocokan ini, semakin besar kemungkinan seseorang akan menunjukkan performa yang baik, memiliki motivasi yang lebih tinggi, dan mencapai kepuasan dalam pekerjaannya.

Teori ini berangkat dari premis bahwa pekerjaan atau kegiatan memiliki tuntutan tertentu, sementara individu memiliki atribut yang berbeda-beda. Apabila atribut tersebut tidak sesuai, maka individu dapat mengalami kesulitan dalam menjalankan tugas atau bahkan kehilangan minat. Sebaliknya, jika kecocokan tercapai, individu akan lebih mampu beradaptasi, menyelesaikan tugas dengan efektif, dan memperoleh pengalaman belajar yang lebih optimal. Dalam konteks kegiatan kolaboratif mahasiswa, \textit{person job fit} sangat relevan karena setiap proyek memiliki kebutuhan spesifik misalnya keterampilan teknis, kemampuan komunikasi, atau preferensi peran tertentu. Mahasiswa yang memiliki atribut sesuai dengan kebutuhan tersebut cenderung lebih mampu berkontribusi secara efektif dalam tim, serta memperoleh pengalaman belajar yang lebih bermakna. Oleh karena itu, sistem rekomendasi yang mampu mengidentifikasi kecocokan antara profil mahasiswa dan karakteristik kegiatan kolaboratif dapat membantu meningkatkan kualitas kolaborasi dan keberhasilan proyek.

\subsection{\textit{Vocational Interests Theory}}

\Textcite{hollandTheoryVocationalChoice1959} menjelaskan bahwa individu memiliki kecenderungan minat dan preferensi aktivitas tertentu yang dapat digunakan untuk memprediksi kecocokan mereka terhadap suatu lingkungan kerja atau aktivitas. Model ini dikenal sebagai RIASEC, yang terdiri dari enam tipe utama yaitu \textit{realistic, investigative, artistic, social, enterprising, and conventional}. Setiap tipe mencerminkan pola minat, aktivitas yang disukai, serta kompetensi yang relatif stabil. Dengan kata lain, semakin tinggi kecocokan antara tipe individu dan karakteristik aktivitas, semakin besar kemungkinan seseorang menunjukkan performa yang baik serta merasa puas dan terlibat. RIASEC juga mencerminkan \textit{patterns of people's interests, preferred activities, competencies, and self-perceptions}, sehingga relevan digunakan untuk memahami bagaimana minat mahasiswa dapat mempengaruhi preferensi mereka terhadap jenis proyek atau kegiatan kolaboratif tertentu.

\subsection{\textit{Rule Based Matching}}

\textit{rule based matching} telah lama digunakan dalam bidang kecerdasan buatan dan sistem pakar. Menurut \Textcite{jacksonClassicalElectrodynamics1999}, \textit{rule based matching} adalah sistem berbasis aturan bekerja dengan prinsip \textit{if then rules}, yaitu aturan yang menghubungkan kondisi tertentu dengan hasil atau aksi yang harus dilakukan. Aturan ini disusun berdasarkan pengetahuan domain yang jelas, sehingga proses pengambilan keputusan bersifat transparan dan mudah dijelaskan. Dalam konteks sistem rekomendasi, \Textcite{burkeHybridRecommenderSystems2002} menjelaskan bahwa \textit{rule based matching} termasuk dalam kategori \textit{knowledge based recommender systems}, yaitu sistem rekomendasi yang memberikan saran berdasarkan pengetahuan eksplisit tentang kebutuhan pengguna dan karakteristik item. Pendekatan ini tidak bergantung pada data historis seperti pada \textit{collaborative filtering}, tetapi menggunakan representasi pengetahuan yang terstruktur, misalnya daftar persyaratan, parameter, atau atribut item yang dapat dievaluasi.

Salah satu kelebihan utama dari \textit{rule based matching} adalah tingkat interpretabilitas yang tinggi. \Textcite{russellArtificialIntelligenceModern2010} menekankan bahwa \textit{rule based matching} sebagai bagian dari \textit{knowledge based systems} memberikan kontrol yang lebih besar terhadap perilaku sistem dibandingkan pendekatan statistik, sehingga memudahkan penjelasan hasil kepada pengguna. \Textcite{burkeHybridRecommenderSystems2002} juga menyoroti bahwa dalam konteks sistem rekomendasi, \textit{knowledge based systems} memungkinkan rekomendasi yang dapat diinterpretasikan dan divalidasi secara langsung oleh pakar domain. 

Namun, \textit{rule based matching} dapat menjadi kurang fleksibel ketika jumlah aturan meningkat atau domain menjadi sangat kompleks. \Textcite{russellArtificialIntelligenceModern2010} menyatakan bahwa sistem berbasis pengetahuan mengalami penurunan efektivitas ketika kompleksitas domain meningkat, karena setiap kemungkinan harus dirumuskan secara eksplisit dalam bentuk aturan. Sebagai respons terhadap keterbatasan tersebut, implementasi modern dari \textit{rule based matching} banyak mengembangkan mekanisme penilaian yang lebih adaptif, seperti penggunaan pembobotan skor afinitas atau \textit{affinity based matching}. 

\subsection{\textit{Affinity based matching}}

\textit{Affinity based matching} merupakan pengembangan dari pendekatan \textit{rule based matching} yang menambahkan mekanisme penilaian kuantitatif untuk menentukan tingkat kecocokan antara dua entitas. Jika \textit{rule based matching} tradisional hanya menghasilkan keluaran biner cocok atau tidak cocok maka \textit{affinity based matching} memberikan \textit{affinity score}, yaitu nilai numerik yang merepresentasikan seberapa besar tingkat kecocokan berdasarkan bobot atribut yang relevan.

Pendekatan ini sejalan dengan apa yang disebut \textit{knowledge based scoring} dalam sistem rekomendasi. \Textcite{burkeHybridRecommenderSystems2002} menjelaskan bahwa \textit{knowledge based scoring} dalam sistem rekomendasi dapat diperluas dengan mekanisme pembobotan untuk mengatasi keterbatasan aturan biner dan menghasilkan rekomendasi yang lebih adaptif terhadap preferensi pengguna. Dengan memberikan bobot pada setiap atribut, sistem dapat menilai tingkat kecocokan secara bertingkat, bukan hanya memenuhi atau melanggar aturan. Konsep ini juga berkaitan erat dengan pendekatan \textit{constraint based recommendation}. \Textcite{felfernigBasicApproachesRecommendation2014} menyatakan bahwa rekomendasi dapat dihitung berdasarkan tingkat pemenuhan serangkaian \textit{constraints}, di mana setiap kendala memiliki tingkat kepentingan yang berbeda. Model ini mendukung penilaian \textit{weighted constraints}, yaitu mekanisme yang memberikan bobot berbeda pada atribut untuk menghasilkan skor relevansi atau kecocokan yang lebih akurat.

Dalam tinjauan mengenai sistem rekomendasi, \Textcite{adomaviciusContextAwareRecommenderSystems2011} menekankan bahwa relevansi antara pengguna dan item tidak harus dilihat secara biner, tetapi dapat dievaluasi melalui pendekatan multidimensional yang menggabungkan berbagai atribut dan bobotnya. Pendekatan ini memperkuat dasar konseptual \textit{affinity based matching} sebagai metode yang menghitung kecocokan secara numerik. Dalam konteks pencocokan mahasiswa dengan kegiatan kolaboratif, pendekatan ini sangat relevan. Atribut seperti minat, keterampilan, pengalaman, preferensi peran, dan tingkat komitmen tidak memiliki pengaruh yang sama. Dengan memberikan bobot pada atribut yang lebih penting, sistem dapat menghasilkan skor afinitas yang lebih representatif terhadap kecocokan aktual antara mahasiswa dan kegiatan. 


\section{\textit{User Centered Design} (UCD)}

\textit{user centered design} (UCD) merupakan pendekatan perancangan sistem yang berfokus pada kebutuhan, tujuan, karakteristik, dan keterbatasan pengguna. Ide utama UCD adalah bahwa desain harus dimulai dari pemahaman tentang manusia sebagai pengguna akhir, bukan dari teknologi itu sendiri. \Textcite{normanDesignEverydayThings1986} menegaskan bahwa sistem yang baik harus dibangun berdasarkan pemahaman mendalam mengenai perilaku, kemampuan kognitif, dan konteks penggunaan pengguna agar teknologi dapat mendukung manusia secara efektif. Pendekatan ini menempatkan pengguna sebagai inti dari proses desain sehingga solusi yang dihasilkan lebih relevan dan usable.

Pendekatan UCD dalam praktik modern distandardisasi melalui ISO 9241-210, yang mendefinisikan \textit{human centred design} (HCD) sebagai kerangka kerja formal untuk merancang sistem interaktif yang berorientasi pada manusia. Meskipun ISO menggunakan istilah \textit{human centred design}, prinsip dan proses yang dijelaskan pada dasarnya identik dengan konsep \textit{user centered design} dalam literatur HCI. ISO 9241-210 menetapkan bahwa proses perancangan harus berfokus pada pemahaman pengguna dan dievaluasi secara iteratif berdasarkan umpan balik pengguna. Standar ini menawarkan kerangka proses yang sistematis sehingga UCD dapat diterapkan secara konsisten dalam pengembangan sistem.

\vspace{20em}

\begin{figure}[h] % pilihan opsi yang disarankan: t = top, b = bottom, h = here
	\centering
  \captionsetup{justification=centering}
    	\includegraphics[width=1\textwidth]{image/UCD.png}
	\caption{UCD method}
	\label{gambar:UCD}
\end{figure}

Berdasarkan gambar \ref{gambar:UCD}, ISO 9241-210\Textcite{ISO924121020102010} menjelaskan proses \textit{user centered design} sebagai berikut.

\begin{enumerate}
\item	\textit{Plan the human centred design process} \\ Pada tahap ini, pengembang menyusun rencana menyeluruh mengenai proses desain yang akan dilakukan. Aktivitas mencakup penentuan tujuan desain, ruang lingkup sistem, metode yang digunakan, sumber daya yang diperlukan, serta strategi pelibatan pengguna. 
\item   \textit{Understand and specify the context of use} \\ Tahap ini berfokus pada pemahaman yang mendalam tentang siapa pengguna sistem, apa tujuan mereka, tugas apa yang harus diselesaikan, serta kondisi lingkungan di mana sistem digunakan. Informasi ini diperoleh melalui analisis pengguna, observasi, wawancara, maupun studi konteks. 
\item   \textit{Specify the user requirements} \\ Pada tahap ini dilakukan pemahaman terhadap konteks penggunaan, yang kemudian menjadi dasar untuk merumuskan kebutuhan pengguna secara lebih spesifik. Kebutuhan ini meliputi kebutuhan fungsional, non-fungsional, batasan penggunaan, preferensi interaksi, serta ekspektasi pengguna terhadap sistem. User requirements ini kemudian digunakan sebagai acuan untuk merancang solusi. 
\item   \textit{Produce design solutions to meet user requirements} \\ Pada tahap ini, pengembang menghasilkan solusi desain berdasarkan kebutuhan pengguna yang telah ditetapkan. Solusi yang dihasilkan dapat berupa sketsa awal, wireframe, model alur (user flow), hingga prototipe interaktif. 
\item   \textit{Evaluate the designs against requirements} \\ Desain yang dihasilkan kemudian dievaluasi dengan melibatkan pengguna untuk menilai sejauh mana desain tersebut memenuhi kebutuhan mereka. Evaluasi dapat dilakukan melalui usability testing, wawancara, observasi, atau metode evaluasi empiris lainnya. 
\end{enumerate}

% \begin{table}[h]
% \centering
% \begin{tabular}{|p{3cm}|p{6cm}|p{3cm}|}
% \hline
% \textbf{Jenis} & \textbf{Definisi} & \textbf{Contoh} \\ \hline

% Innovation Platform &
% Platform yang menjadi fondasi teknologi bagi pihak ketiga untuk membangun aplikasi, layanan, atau produk pelengkap. &
% iOS, Android, Windows. \\ \hline

% Transaction Platform &
% Platform yang bertindak sebagai perantara untuk mempertemukan dua atau lebih pihak yang melakukan transaksi barang, jasa, atau informasi. &
% Gojek, Tokopedia, Uber. \\ \hline

% Hybrid Platform &
% Penggabungan innovation dan transaction platform dalam satu ekosistem; memungkinkan perusahaan mengendalikan teknologi serta distribusi. &
% Google (Android + Play Store), Apple (iOS + App Store). \\ \hline

% \end{tabular}
% \caption{Tabel jenis platform}
% \label{tbl:jenis_platform}
% \end{table}

% Perbedaan utama antara kedua tipe utama platform terletak pada aktivitas utama yang mereka fasilitasi. \textit{Innovation platforms} berfokus pada penciptaan ekosistem inovasi melalui komplementor yang memperluas fungsi platform. Sebaliknya, \textit{transaction platforms} memusatkan aktivitas pada transaksi dan pertukaran antar pengguna. Dalam praktiknya, perusahaan teknologi global cenderung mengadopsi hybrid model karena strategi ini memungkinkan mereka memanfaatkan data, jaringan pengguna, dan komplementor sekaligus sehingga memberikan keunggulan kompetitif yang sulit ditanding

% \section{\textit{User Centered Design}}

% \textit{User Centered Design} (UCD) adalah pendekatan perancangan sistem yang menempatkan pengguna sebagai pusat dari seluruh proses desain, mulai dari pemahaman masalah hingga evaluasi solusi. Menurut \Textcite{normanDesignEverydayThings2013} UCD berangkat dari prinsip bahwa keberhasilan suatu produk sangat ditentukan oleh kesesuaiannya dengan cara manusia berpikir, bertindak, memproses informasi, serta menyelesaikan tugas. Oleh karena itu, UCD menekankan perlunya memahami perilaku, tujuan, hambatan, motivasi, dan konteks penggunaan sebelum merancang solusi apa pun.

% \Textcite{gulliksenKeyPrinciplesUsercentred2003} juga menekankan bahwa UCD harus melibatkan proses iteratif di mana desain dan evaluasi dilakukan berulang kali hingga mencapai tingkat usability yang optimal. Mereka menyatakan bahwa proses UCD akan menghasilkan sistem yang lebih efektif dan efisien karena keputusan desain selalu divalidasi berdasarkan perspektif pengguna, bukan sekadar preferensi teknis pengembang.

% Sejalan dengan pandangan tersebut, \Textcite{sharpInteractionDesign2015} menjelaskan bahwa UCD mencakup empat tahapan utama yang membentuk siklus berulang dan saling terkait. Keempat tahapan tersebut meliputi;

% \begin{enumerate}[a)]
% \item Memahami pengguna dan konteks penggunaan, yaitu aktivitas untuk mengidentifikasi siapa pengguna, apa tujuan mereka, tugas yang dilakukan, serta lingkungan di mana sistem digunakan melalui observasi, wawancara, atau studi aktivitas.
% \item Menetapkan kebutuhan dan persyaratan pengguna, dari temuan awal diterjemahkan menjadi kebutuhan fungsional, non-fungsional, dan kebutuhan pengalaman pengguna yang menjadi dasar perancangan.
% \item Menghasilkan alternatif solusi desain, berupa pembuatan sketsa, \textit{wireframe}, dan prototipe awal yang mengeksplorasi berbagai kemungkinan solusi yang berpotensi memenuhi kebutuhan pengguna.
% \item Mengevaluasi desain bersama pengguna, yaitu proses menilai efektivitas dan kenyamanan desain melalui usability testing, think-aloud, atau metode evaluatif lainnya untuk memastikan kesesuaiannya dengan kebutuhan pengguna.
% \end{enumerate}






% \subsection{Gambar}
% Contoh gambar dapat dilihat pada Gambar \ref{gambar:jaringan}. Gambar dan judulnya diposisikan di tengah. Nomor gambar tidak diakhiri tanda titik. Gambar tersebut dibuat menggunakan aplikasi draw.io dan disimpan ke format PNG setelah dengan zoom setting pada angka 300\%. Ukuran gambar yang ditampilkan dapat diatur dengan mengubah nilai \textit{width} dalam sintaks \textit{includegraphics}.



% Gambar umumnya tidak jelas atau kabur jika gambar tersebut:
% \begin{enumerate}[a.]
%   \item diperoleh dari hasil cropping pada suatu halaman buku atau situs web;
%   \item hasil pembesaran gambar yang gambar aslinya sebenarnya berukuran kecil; atau
%   \item disimpan dalam resolusi kecil
% \end{enumerate}
% Ketidakjelasan gambar ini dapat dilihat pada garis-garis diagram yang tidak tegas dan tulisan-tulisan dalam gambar yang tampak kabur dan kurang jelas terbaca.

% Untuk mendapatkan gambar yang tidak kabur (\textit{blur}), langkah-langkah berikut dapat digunakan:
% \begin{enumerate}[(a)]
% \item Gambar yang didapat di suatu pustaka atau referensi sebaiknya digambar ulang, misalnya menggunakan PowerPoint, Canva, Figma, draw.io, atau yang lainnya.
% \item Jika diagram atau ilustrasi digambar menggunakan draw.io, saat gambar disimpan ke format PNG atau JPG (\textit{export as}), lakukan \textit{zoom} ke minimal 300\% (\textit{the default value is} 100\%). 
% \item Jika diagram digambar dengan menggunakan PowerPoint, gambar dapat langsung di-\textit{copy-paste} ke Word.
% \end{enumerate}

% \subsection{Tabel}
% Tabel ada dua jenis, yaitu tabel yang bisa termuat dalam satu halaman dan tabel yang sangat panjang sehingga tidak muat dalam satu halaman.
% \subsubsection{Tabel yang Muat dalam Satu Halaman}
% Contoh tabel dapat dilihat pada Tabel \ref{tbl:harga1} dan \ref{tbl:harga2}. Tabel dan judulnya dibuat rata kiri dan judul tabel diletakkan di atas tabel. Usahakan tabel dapat ditulis dalam satu halaman, tidak terpotong ke halaman berikutnya.

% \begin{table}[t] % pilihan opsi yang disarankan: t = top, b = bottom, h = here
%   \begin{tabular}{ | p{2cm} | p{2cm} | p{3cm} |}
% 	\hline
% 	Nama 	& Satuan 		& Harga \\
% 	\hline
% 	Buku 	& Exemplar	& 25000 \\
% 	Komputer	& Unit		& 2500000 \\
% 	Pensil	& Buah		& 118900 \\
% 	\hline
% 	\end{tabular}
% \caption{Tabel harga bahan pokok}
% \label{tbl:harga1}
% \end{table}

% \begin{table}[t] % pilihan opsi yang disarankan: t = top, b = bottom, h = here
% 	\begin{tabular}{ | l | c | r | }
% 	\hline
% 	Nama 	& Satuan 		& Harga \\
% 	\hline
% 	Buku 	& Exemplar	& 25000 \\
% 	Komputer	& Unit		& 2500000 \\
% 	Pensil	& Buah		& 118900 \\
% 	\hline
% 	\end{tabular}
% \caption{Tabel harga bahan sekunder}
% \label{tbl:harga2}
% \end{table}

% \subsubsection{Tabel yang Sangat Panjang}
% Jika tabel terlalu panjang sehingga tidak muat dalam satu halaman, gunakan paket 
% \textit{longtable} untuk membuat tabel yang dapat terpotong ke halaman berikutnya, 
% seperti pada Tabel \ref{tbl:longtable1}.

% \begin{longtable}{@{\extracolsep{\fill}} l c r r}
% \caption{Comprehensive Data Table Example}\label{tbl:longtable1} \\
% \toprule
% \textbf{ID} & \textbf{Name} & \textbf{Score} & \textbf{Rank} \\
% \midrule
% \endfirsthead

% \caption*{Comprehensive Data Table Example (lanjutan)} \\
% \toprule
% \textbf{ID} & \textbf{Name} & \textbf{Score} & \textbf{Rank} \\
% \midrule
% \endhead

% \midrule
% \multicolumn{4}{r}{\textit{Bersambung ke halaman berikutnya}} \\
% \bottomrule
% \endfoot

% \bottomrule
% \endlastfoot

% % === Table Data ===
% 1 & Alice Smith & 89 & 5 \\
% 2 & Bob Johnson & 93 & 3 \\
% 3 & Carol Davis & 95 & 2 \\
% 4 & Daniel Wilson & 88 & 6 \\
% 5 & Eve Thompson & 97 & 1 \\
% 6 & Frank Brown & 85 & 7 \\
% 7 & Grace Lee & 91 & 4 \\
% 8 & Henry Miller & 80 & 9 \\
% 9 & Irene Garcia & 83 & 8 \\
% 10 & Jack Robinson & 78 & 10 \\
% % Repeat with more rows to make it long
% 11 & Kevin Harris & 76 & 11 \\
% 12 & Laura Martin & 75 & 12 \\
% 13 & Michael Clark & 74 & 13 \\
% 14 & Natalie Lewis & 73 & 14 \\
% 15 & Olivia Walker & 72 & 15 \\
% 16 & Peter Hall & 71 & 16 \\
% 17 & Quinn Allen & 70 & 17 \\
% 18 & Rachel Young & 69 & 18 \\
% 19 & Samuel King & 68 & 19 \\
% 20 & Tina Wright & 67 & 20 \\
% 21 & Uma Scott & 66 & 21 \\
% 22 & Victor Green & 65 & 22 \\
% 23 & Wendy Adams & 64 & 23 \\
% 24 & Xavier Nelson & 63 & 24 \\
% 25 & Yolanda Carter & 62 & 25 \\
% 26 & Zachary Perez & 61 & 26 \\
% 27 & Amelia Baker & 60 & 27 \\
% 28 & Benjamin Rivera & 59 & 28 \\
% 29 & Charlotte Rogers & 58 & 29 \\
% 30 & David Murphy & 57 & 30 \\
% 31 & Ethan Cooper & 56 & 31 \\
% 32 & Fiona Reed & 55 & 32 \\
% 33 & George Bailey & 54 & 33 \\
% 34 & Hannah Cox & 53 & 34 \\
% 35 & Isaac Howard & 52 & 35 \\
% 36 & Julia Ward & 51 & 36 \\
% 37 & Kyle Flores & 50 & 37 \\
% 38 & Lily Bell & 49 & 38 \\
% 39 & Mason Sanders & 48 & 39 \\
% 40 & Nora Patterson & 47 & 40 \\
% 41 & Owen Ramirez & 46 & 41 \\
% 42 & Penelope Torres & 45 & 42 \\
% 43 & Quentin Foster & 44 & 43 \\
% 44 & Rebecca Gonzales & 43 & 44 \\
% 45 & Sebastian Bryant & 42 & 45 \\
% 46 & Taylor Alexander & 41 & 46 \\
% 47 & Ursula Russell & 40 & 47 \\
% 48 & Vincent Griffin & 39 & 48 \\
% 49 & William Diaz & 38 & 49 \\
% 50 & Zoe Simmons & 37 & 50 \\
% % (You can easily extend this list to hundreds of rows)
% \end{longtable}

% \subsubsection{Rumus}
% Contoh rumus matematika dapat ditulis seperti pada Persamaan \ref{eq:contoh1} di bawah ini. 
% Penomoran persamaan diletakkan di sebelah kanan, dan rumus ditulis dalam mode \textit{display math}.
% \begin{equation}
% E = mc^2
% \label{eq:contoh1}
% \end{equation}

% Contoh lain penulisan rumus matematika yang lebih kompleks dapat ditulis seperti pada Persamaan \ref{eq:rumus2}.

% \begin{align}
% f(x) &= ax^2 + bx + c \\
% f'(x) &= \frac{d}{dx}(ax^2 + bx + c) \notag \\ % tidak menampilkan nomor pada baris ini
%       &= 2ax + b \label{eq:rumus2}
% \end{align}

% Jika rumus terlalu panjang untuk ditulis dalam satu baris, gunakan lingkungan \textit{multline} seperti pada Persamaan \ref{eq:rumus3} di bawah ini.
% \begin{multline} 
% y = a_0 + a_1x + a_2x^2 + a_3x^3 + a_4x^4 + a_5x^5 + a_6x^6 + a_7x^7 \\
%     + a_8x^8 + a_9x^9 + a_{10}x^{10} \label{eq:rumus3}
% \end{multline}

% Jika ada penurunan rumus yang terdiri dari beberapa baris, namun tidak memerlukan penomoran pada setiap baris, gunakan lingkungan \textit{align*}, misalnya:

% \begin{align*} 
% S &= \sum_{i=1}^{n} i^2 \\
%   &= 1^2 + 2^2 + 3^2 + \cdots + n^2 \\
%   &= \frac{n(n + 1)(2n + 1)}{6}
% \intertext{Contoh lainnya adalah rumus untuk mencari nilai rata-rata fungsi $f(x)$ pada interval $[p, q]$:}
% \bar{f} &= \frac{1}{q - p} \int_{p}^{q} f(x) \, dx \\
%         &= \frac{1}{q - p} \int_{p}^{q} (ax^2 + bx + c) \, dx \\
%         &= \frac{1}{q - p} \left[ \frac{a}{3}x^3 + \frac{b}{2}x^2 + cx \right]_p^q \\
%         &= \frac{a(q^3 - p^3)}{3(q - p)} + \frac{b(q^2 - p^2)}{2(q - p)} + c \label{eq:rumus4}
% \end{align*}



% \subsection{Algoritma, Pseudocode, atau Kode}
% Contoh penulisan algoritma atau pseudocode dapat ditulis seperti pada Kode \ref{alg:contoh1} di bawah ini. 
% Gunakan paket \textit{listings} untuk menulis source code dalam bahasa pemrograman tertentu, seperti pada Kode \ref{lst:contoh1}. 


% % -- Example of pseudocode and source code listing --
% % -- Gunakan minipage agar listing tidak terpotong ke halaman berikutnya --
% \begin{minipage}{\textwidth} 
% \begin{lstlisting}[frame=lines, captionpos=t, caption={Contoh pseudocode}, label={alg:contoh1}]
% ALGORITHM HelloWorld
%    PRINT "Hello, World!"
% END ALGORITHM
% \end{lstlisting}
% \end{minipage}

% \begin{minipage}{\textwidth}
% \begin{lstlisting}[language=Python, frame=single, caption={Contoh source code Python}, captionpos=t, label={lst:contoh1}]
% def hello_world():
%     print("Hello, World!")       
% hello_world()
% \end{lstlisting}
% \end{minipage}


% \section{Beberapa Kesalahan Penulisan yang Sering Terjadi}
% \subsection{Penggunaan Kata "di mana" atau "dimana"}
% Banyak yang menuliskan kata "di mana" atau "dimana" sebagai pengganti kata "which" dalam bahasa Inggris. 
% Padahal, penggunaan kata "di mana" atau "dimana" tidak tepat dalam konteks tersebut. Demikian juga untuk kata serupa, misalnya "yang mana".
% Kata "di mana" atau "dimana" ini harus diganti dengan kata lain, seperti "dengan", "tempat", "yang", dan sebagainya tergantung kalimatnya.
% Penjelasan lengkap dapat dilihat pada \autocite{BPBI}.

% \subsection{Penggunaan Kata "sedangkan" dan "sehingga"}

% \begin{table}[t]
%   \begin{tabular}{|c|l|l|}
%   \hline
%   Kata & Salah & Benar \\ \hline
%   sedangkan & \begin{tabular}[c]{@{}c@{}}Sedangkan sistem lama masih\\ digunakan oleh banyak pengguna.\end{tabular} & \begin{tabular}[c]{@{}c@{}}Sistem lama masih digunakan\\ oleh banyak pengguna,\\ sedangkan sistem baru belum siap.\end{tabular} \\ \hline
%   sehingga & \begin{tabular}[c]{@{}c@{}}Sehingga sistem lama masih\\ digunakan oleh banyak pengguna.\end{tabular} & \begin{tabular}[c]{@{}c@{}}Sistem lama masih digunakan\\ oleh banyak pengguna sehingga\\ sistem baru belum siap.\end{tabular} \\ \hline
%   \end{tabular}
%   \caption{Contoh penggunaan kata "sedangkan" dan "sehingga"}
%   \label{tbl:sedangkan_sehingga}
% \end{table}

% Kata "sedangkan" dan "sehingga" adalah kata hubung atau konjungsi. 
% Konjungsi adalah kata atau ungkapan yang menghubungkan satuan bahasa 
% (kata, frasa, klausa, dan kalimat). 
% Konjungsi dapat dibagi menjadi konjungsi intrakalimat dan antarkalimat.  
% Kata "sedangkan" menghubungkan dua klausa yang bersifat kontrasif, 
% sedangkan "sehingga" menghubungkan dua klausa yang bersifat kausal. 
% Dalam ragam formal, kata hubung “sedangkan” dan “sehingga” hanya dapat digunakan 
% sebagai konjungsi intrakalimat sehingga kedua konjungsi itu \textbf{tidak dapat diletakkan pada awal kalimat}.
% Selain itu, penggunaan kata "sedangkan" harus didahului oleh koma (,), sedangkan kata "sehingga" tidak perlu didahului oleh koma (,).
% Contoh penggunaan yang benar dan salah dapat dilihat pada Tabel \ref{tbl:sedangkan_sehingga}.


% \subsection{Penggunaan Istilah yang Tidak Baku}
% Ada beberapa istilah yang sering digunakan dalam pembicaraan sehari-hari, tetapi tidak baku dalam penulisan ilmiah.
% Beberapa istilah tersebut antara lain:
% \begin{enumerate}
%   \item analisa $\rightarrow$ analisis
%   \item eksisting atau existing $\rightarrow$ yang ada atau saat ini
%   \item bisnis proses $\rightarrow$ proses bisnis
%   \item user $\rightarrow$ pengguna
%   \item system $\rightarrow$ sistem
%   \item database $\rightarrow$ basis data
%   \item aktifitas $\rightarrow$ aktivitas
%   \item efektifitas $\rightarrow$ efektivitas
%   \item sosial media $\rightarrow$ media sosial
% \end{enumerate}

% \subsection{Pemisah Desimal dan Ribuan}
% Tanda pemisah desimal dalam bahasa Indonesia adalah tanda koma, contoh:
% \begin{enumerate}
%   \item (Salah) Akurasi naik menjadi 50.6\% 
%   \item (Benar) Akurasi naik menjadi 50,6\% 
% \end{enumerate}

% \subsection{Daftar atau \textit{List}}
% Ada beberapa aturan penulisan daftar atau \textit{list} yang perlu diperhatikan, antara lain:
% \begin{enumerate}[a)]
% \item Jika memungkinkan, hindari penggunaan “bullet points” atau sejenisnya. Sebaiknya, gunakan angka (1, 2, 3, ...) atau huruf (a, b, c, …). Dengan demikian, pembaca dapat dengan mudah melihat jumlah \textit{item} atau \textit{list}. 
% \item Jika dalam daftar hanya ada satu item, tidak perlu menggunakan nomor urut.
% \item Penjelasan atau deskripsi suatu item sebaiknya menyatu dengan judul item tersebut, tidak berbeda halaman. Contoh yang salah: judul item ada di halaman 10, namun deskripsinya di halaman 11. Sebaiknya pindahkan judul tersebut ke halaman 11.
% \item Jika penjelasan atau deskripsi suatu item cukup panjang, misalnya lebih dari 1 halaman atau terdiri atas beberapa paragraf, sebaiknya setiap item tersebut dijadikan judul subbab, kecuali jika level subbab sudah mencapai level 4. 
% \end{enumerate}

% \subsection{Penggunaan Kata "masing-masing" dan "setiap"}
% Kata “masing-masing” digunakan di belakang kata yang diterangkan, misalnya 
% "Setiap proses menggunakan algoritma masing-masing". Kata “tiap-tiap” atau “setiap”
% ditempatkan di depan kata yang diterangkan, misalnya
% "Setiap proses menggunakan algoritma tertentu".

% ==========================================
% BAB I PENDAHULUAN
% ==========================================
\chapter{PENDAHULUAN}
\label{chap:pendahuluan}
% --- Latar Belakang ---
\section{Latar Belakang}
Mahasiswa merupakan individu yang berada pada fase penting dalam perkembangan akademik dan profesional. Pada tahap ini, mereka bukan hanya dituntut memahami teori, tetapi juga harus mampu menerapkannya melalui pengalaman nyata kegiatan kolaboratif seperti proyek, kompetisi, penelitian, magang, maupun kegiatan kolaboratif lainnya. Menurut \Textcite{deweyEXPERIENCEEDUCATION1938}, pembelajaran akan menjadi bermakna jika mahasiswa terlibat langsung dalam aktivitas yang memberikan pengalaman konkrit, karena pengalaman tersebut membangun kemampuan berpikir dan memecahkan masalah secara lebih mendalam.

Namun, kesempatan untuk memperoleh pengalaman tersebut tidak selalu mudah diakses secara sistematis. Informasi peluang kegiatan kolaboratif bagi mahasiswa umumnya tersebar di berbagai platform seperti media sosial, grup percakapan, dan jaringan informal. Menurut Barry Schwartz (2004), kondisi informasi yang melimpah tetapi tidak terorganisasi dapat memicu \textit{choice overload}, yaitu kesulitan individu dalam mengambil keputusan meskipun banyak opsi tersedia. Dalam konteks mahasiswa, situasi ini dapat membuat proses pemilihan kegiatan kolaboratif menjadi tidak terarah karena mereka tidak memiliki mekanisme untuk membandingkan kecocokan antara kebutuhan kegiatan kolaboratif dan profil diri. Konsekuensinya, mahasiswa dapat kesulitan mengidentifikasi peluang yang benar-benar relevan dengan minat, kemampuan, dan tujuan pengembangan diri mereka, sehingga potensi belajar melalui pengalaman tidak dapat dimanfaatkan secara optimal.

Mahasiswa pada dasarnya memiliki akses terhadap berbagai platform pencarian peluang, termasuk layanan seperti LinkedIn, Glints, atau Kalibrr. Namun, platform-platform tersebut umumnya dirancang untuk kebutuhan rekrutmen profesional dan lowongan pekerjaan formal, sehingga tidak secara spesifik mengakomodasi karakteristik dan kebutuhan mahasiswa dalam mencari kegiatan kolaboratif atau pengalaman belajar berbasis praktik. Hal ini membuat mahasiswa, terutama yang masih berada pada tahap awal perkembangan karier, kesulitan menemukan peluang yang benar-benar sesuai dengan tingkat kemampuan, minat belajar, serta tujuan pengembangan diri mereka. Selain itu, sistem rekomendasi pada platform-platform tersebut belum dirancang untuk memberikan pemetaan kecocokan antara profil mahasiswa dengan karakteristik kegiatan kolaboratif yang bersifat akademik atau kolaboratif, sehingga proses pencarian peluang sering kali tidak memberikan arahan yang jelas. 

Kondisi tersebut semakin menantang karena mahasiswa tidak selalu mengetahui apakah keterampilan, preferensi, atau gaya kolaborasi mereka cocok dengan peran yang ditawarkan dalam sebuah kegiatan kolaboratif. Menurut \Textcite{schonEducatingReflectivePractitioner1987}, individu perlu memahami profil dirinya baik kemampuan maupun cara bekerja agar dapat memilih aktivitas profesional yang selaras dengan dirinya. Tanpa adanya platform yang dapat membantu mahasiswa memetakan profil tersebut terhadap kebutuhan kegiatan kolaboratif, mereka cenderung memilih peluang secara \textit{trial and error}. Proses ini tidak hanya memakan waktu, tetapi juga berpotensi menghasilkan ketidakcocokan peran, tingkat komitmen yang rendah, atau bahkan kegagalan dalam menyelesaikan kegiatan kolaboratif secara optimal.

% \begin{enumerate}
% \item	Kondisi atau situasi topik yang dibahas beserta permasalahannya, misalnya tentang pengelolaan informasi di puskesmas daerah pedesaan dan masalah yang dihadapi.
% \item	Berbagai solusi yang teDlah diterapkan atau solusi yang tersedia dan memungkinkan untuk diterapkan untuk menyelesaikan masalah tersebut.
% \end{enumerate}

% --- Rumusan Masalah ---
% \section{Rumusan Masalah}
% Rumusan Masalah berisi masalah utama yang dibahas dalam tugas akhir. Rumusan masalah yang baik memiliki struktur sebagai berikut:
% \begin{enumerate}
% \item	Pokok persoalan dari kondisi atau situasi yang ada saat ini. Dengan kata lain, jelaskan kelemahan atau kekurangan dari kondisi, situasi, atau solusi yang dijelaskan pada latar belakang. Ini merupakan pokok rumusan masalah.
% \item	Elaborasi lebih lanjut urgensi penyelesaian masalah tersebut (mengapa penting untuk diselesaikan dan akibat yang dapat terjadi jika tidak diselesaikan).
% \item	Usulan singkat terkait dengan solusi yang ditawarkan untuk menyelesaikan persoalan.
% Penting untuk diperhatikan bahwa persoalan yang dideskripsikan pada subbab ini akan dipertanggungjawabkan di bab Evaluasi (apakah terselesaikan atau tidak).
% \end{enumerate}
\section{Rumusan Masalah}
Berdasarkan Latar belakang di atas, maka rumusan masalah dari tugas akhir ini adalah bagaimana mengembangkan sistem yang dapat membantu mahasiswa dalam menemukan dan memilih kegiatan kolaboratif yang relevan dengan profil mereka.

% Berdasarkan latar belakang tersebut, maka rumusan masalah dalam tugas akhir ini adalah sebagai berikut:
% \begin{enumerate}
% \item	Bagaimana merancang konsep dan model sistem platform kolaborasi mahasiswa yang mampu menghubungkan pengguna berdasarkan minat, keahlian, dan kebutuhan kolaborasi mereka?
% \item	Bagaimana merancang mekanisme matching dan verifikasi pengguna agar proses pencarian partner menjadi lebih tepat, cepat, dan kredibel?
% \item	Bagaimana merancang fitur dan alur interaksi dalam platform yang dapat memfasilitasi pembentukan, komunikasi, dan pengelolaan kolaborasi mahasiswa secara berkelanjutan?
% \end{enumerate}

% --- Tujuan ---
\section{Tujuan}
% Tuliskan tujuan utama dan/atau tujuan detail yang akan dicapai dalam pelaksanaan tugas akhir. Fokuskan pada hasil akhir yang ingin diperoleh setelah tugas akhir diselesaikan, terkait dengan penyelesaian persoalan pada rumusan masalah. Penting untuk diperhatikan bahwa tujuan yang dideskripsikan pada subbab ini akan dipertanggungjawabkan di akhir pelaksanaan tugas akhir apakah tercapai atau tidak. Tuliskan kriteria keberhasilan tugas akhir ini.
Berdasarkan rumusan masalah di atas, tujuan dari tugas akhir ini adalah mengembangkan sistem yang dapat membantu mahasiswa dalam menemukan dan memilih kegiatan kolaboratif yang relevan dengan profil mereka.

% Berdasarkan Rumusan masalah tersebut, maka tujuan dari tugas akhir ini adalah sebagai berikut:
% \begin{enumerate}
% \item	Untuk merancang konsep dan model sistem platform kolaborasi mahasiswa yang mampu menghubungkan pengguna berdasarkan minat, keahlian, dan kebutuhan kolaborasi mereka
% \item	Untuk merancang mekanisme matching dan verifikasi pengguna agar proses pencarian partner menjadi lebih tepat, cepat, dan kredibel
% \item	Untuk memerancang fitur dan alur interaksi dalam platform yang dapat memfasilitasi pembentukan, komunikasi, dan pengelolaan kolaborasi mahasiswa secara berkelanjutan
% \end{enumerate}

% --- Batasan Masalah ---
\section{Batasan Masalah}
% Tuliskan batasan-batasan yang diambil dalam pelaksanaan tugas akhir. Batasan ini dapat dihindari (bersifat opsional, tidak perlu ada) jika topik atau judul tugas akhir dibuat cukup spesifik.
Batasan masalah dari tugas akhir ini adalah sebagai berikut.
\begin{enumerate}
\item	Sistem difokuskan untuk kebutuhan mahasiswa sebagai pengguna utama, khususnya dalam konteks pencarian kegiatan kolaboratif.
\item	Profil mahasiswa yang digunakan dalam proses rekomendasi dibatasi pada aspek tertentu, meliputi minat, keterampilan yang dikuasai, preferensi peran, dan parameter yang relevan.
\end{enumerate}

% --- Metodologi Pengerjaan TA ---
% \section{Metodologi}
% Tuliskan semua tahapan yang akan dilalui selama pelaksanaan tugas akhir. Tahapan ini spesifik untuk menyelesaikan persoalan tugas akhir. Khusus untuk penyusunan proposal ini, jelaskan secara detail:
% \begin{enumerate}
% \item	Tahapan investigasi pengumpulan fakta di latar belakang untuk merumuskan masalah.
% \item	Langkah-langkah pencarian, pengelompokan, dan penapisan literatur atau sumber informasi untuk mengumpulkan informasi yang relevan tentang topik yang diangkat, termasuk teori (konsep atau teori apa saja yang perlu dicari), hal-hal yang telah dicapai oleh orang lain (cara mencari dan kata kuncinya), dan berbagai informasi pendukung, untuk mencari solusi terhadap masalah yang dibahas. Gunakan metodologi yang tepat dalam menggali informasi dan dokumentasikan prosesnya (termasuk rekaman wawancara atau survei) di dalam Lampiran, termasuk tautan ke video atau foto. Hasil penggalian informasi ini akan dijelaskan secara sistematis di Bab II Studi Literatur.
% \end{enumerate}
\section{Metodologi}
Tugas akhir ini menggunakan model \textit{software development life cycle} (SDLC) \textit{waterfall}, sebagaimana dijelaskan oleh Ian Sommerville dalam \textit{Software Engineering} (10th ed.). Model \textit{waterfall} dipilih karena memberikan alur kerja yang terstruktur, terdokumentasi, dan linear, sehingga sesuai untuk proyek Tugas Akhir ini yang memerlukan pemisahan modul antar anggota kelompok, namun tetap harus terintegrasi pada tahap akhir.

Selain itu, proses desain dan evaluasi pada tahap \textit{waterfall} menggunakan pendekatan \textit{user centered design} untuk memastikan platform kolaborasi mahasiswa benar-benar sesuai kebutuhan pengguna. Berikut penjelasan tiap tahap \textit{waterfall} beserta konteks implementasinya pada sistem yang dikembangkan. 

\begin{figure}[h] % pilihan opsi yang disarankan: t = top, b = bottom, h = here
	\centering
  \captionsetup{justification=centering}
    	\includegraphics[width=1\textwidth]{image/waterfall.png}
	\caption{waterfall method}
	\label{gambar:waterfall}
\end{figure}

berdasarkan gambar \ref{gambar:waterfall}, alur SDLC yang digunakan pada pengerjaan tugas akhir ini adalah sebagai berikut.

\begin{enumerate}
\item	\textit{Requirement Definition} \\ Pada tahap ini dilakukan proses pengumpulan dan analisis kebutuhan pengguna dan sistem. Kegiatan meliputi penyebaran survei dan observasi terkait pengalaman mahasiswa dalam mencari tim dan menentukan role. Seluruh temuan kemudian dirumuskan dalam bentuk kebutuhan fungsional dan non-fungsional, yang disusun ke dalam dokumen \textit{software requirement specification} (SRS) sebagai fondasi seluruh tahap berikutnya.
\item   \textit{System and Software Design} \\ Tahap ini menghasilkan rancangan sistem secara menyeluruh, mulai dari arsitektur perangkat lunak, diagram UML, hingga desain antarmuka. Prinsip \textit{user centered design} (UCD) digunakan untuk memastikan rancangan fitur sesuai dengan kebutuhan pengguna nyata berdasarkan temuan pada tahap sebelumnya. Fase ini mencakup penyusunan \textit{use case, flow diagram, class diagram}, serta prototipe wireframe untuk modul \textit{people to people, people to project, and team formation}. Pada tahap ini juga dirancang logika \textit{affinity based matching}, yaitu kerangka rekomendasi berdasarkan preferensi kontribusi mahasiswa.
\item   \textit{Implementation and Unit Testing} \\ Implementasi sistem dilakukan berdasarkan desain yang telah disepakati. Pada proyek ini, setiap anggota kelompok bertanggung jawab untuk mengembangkan satu modul berbeda, yaitu modul \textit{people to people} (Ahmad Fawwazi, 18222117), modul \textit{people to project} (Muhammad Faishal Putra, 18222129) dan modul \textit{team formation} (Muhammad Faishal Firdaus, 18222136). Setiap modul dikembangkan secara independen namun tetap mengikuti standar desain yang sama agar mudah diintegrasikan. Setelah implementasi tiap modul selesai, dilakukan \textit{unit testing} untuk memastikan setiap fungsi berjalan sesuai spesifikasi tanpa adanya kesalahan logika dasar.
\item   \textit{Integration and System Testing} \\ Tahap ini menggabungkan ketiga modul menjadi satu platform yang utuh. Proses integrasi mencakup penyatuan alur data, sinkronisasi API atau fungsi internal, serta penyamaan format keluaran, terutama antara berbagai modul. 
\item   \textit{Operation and Maintenance} \\ Tahap akhir dalam model \textit{waterfall} mencakup proses pemeliharaan sistem setelah pengujian. Kegiatan dalam fase ini meliputi perbaikan bug, penyempurnaan alur interaksi, serta pengoptimalan fitur berdasarkan umpan balik dari proses \textit{usability testing}. 
\end{enumerate}